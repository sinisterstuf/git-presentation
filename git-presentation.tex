%        File: git-presentation.tex
%     Created: Wed Oct 17 12:00 PM 2012 NDT
%
% !TEX program = pdflatex

\documentclass{beamer}

\mode<presentation>
{
  \usetheme{Warsaw}
  \usecolortheme{seahorse}
  \usecolortheme{rose}
  \useinnertheme{rectangles}
  \setbeamercovered{transparent}
}

\usepackage[UKenglish]{babel}
\usepackage[utf8]{inputenc}
\usepackage{lmodern}
\usepackage[T1]{fontenc}
\usepackage{hyperref}
\usepackage{color}
\usepackage{graphicx}

\graphicspath{{screenshots/}}

\setlength\fboxsep{0.5pt}
\setlength\fboxrule{0.5pt}

\hypersetup{
  pdffitwindow=true,%
  pdftitle={Git \& Eclipse},%
  pdfauthor={Si\^on le Roux},%
  colorlinks=true,%
  pdfmenubar=false,%
  pdftoolbar=false,%
  bookmarks=false,%
  unicode=true,%
  linkcolor=black,%
  urlcolor=cyan
}

\title[Git \& Eclipse]{Collaborative Development}
\subtitle{Using git and Eclipse with the EGit and Mylyn plugins.}

\author{Si\^on~Le~Roux}

%\institute{NBIC}
\date{\today}

\subject{Software Development}

%Show the ToC at the beginning of every section and sub-section
\AtBeginSection[]
{
  \begin{frame}{Overview}
    \tableofcontents[currentsection,currentsubsection]
  \end{frame}
}

\AtBeginSubsection[]
{
  \begin{frame}{Overview}
    \tableofcontents[currentsection,currentsubsection]
  \end{frame}
}

% If you wish to uncover everything in a step-wise fashion, uncomment
% the following command: 

%\beamerdefaultoverlayspecification{<+->}

\begin{document}

\begin{frame}
  \titlepage
\end{frame}

\section{Introduction}
\begin{frame}[<+->]
  \frametitle{Introduction}
  \begin{itemize}
	\item Why are we doing this?
	\item Why git and Eclipse?
	  \begin{itemize}
		\item \textbf{Git}: best VCS\footnote{\href{http://git-scm.com/about}{git-scm.com/about}}
		\item free project hosting on \textbf{GitHub}
		\item does \textbf{distributed} Version Control
		\item \textbf{Eclipse}: great Java IDE
		\item \textbf{plug-in based} architecture
	  \end{itemize}
	\item Can we get a copy of this tutorial?\footnote{\href{http://sionleroux.com/files/git-presentation.pdf}{sionleroux.com/files/git-presentation.pdf}}
	\item What do we aim to achieve?
  \end{itemize}
\end{frame}

\section{Set-up}
\subsection{Git}
\begin{frame}[<+->]
  \frametitle{Installing Git}
  \begin{block}{GNU/Linux}
	Install git from your package manager. On Debian based systems, that's: \texttt{\# aptitude install git}
  \end{block}
  \begin{block}{Windows}
	\begin{enumerate}
	  \item Download the \emph{Git for Windows} installer from \href{http://msysgit.github.com}{msysgit.github.com}
	  \item Follow the installation instructions following the installer's recommended ``safe'' options.
	\end{enumerate}
  \end{block}\pause
  You can now type \textbf{\texttt{git}} at the command prompt, and access the Git GUI on Windows.
\end{frame}

\begin{frame}[<+->]
  \frametitle{Setting up Git}
	Follow instructions at \href{https://help.github.com/articles/set-up-git}{help.github.com/articles/set-up-git}

  	In summary:\pause
  \begin{enumerate}
	\item If you don't have one, create an account at \href{http://github.com}{github.com}
	\item Generate a private-public ssh-keypair
	\item Add the public key on GitHub
	\item Set your name and e-mail in the global git config
  \end{enumerate}
  \begin{block}{Setting name and e-mail}
	\texttt{git config \--\--global user.name 'John Doe'}

	\texttt{git config \--\--global user.email 'johndoe@gmail.com'}
  \end{block}\pause
  You can learn git at \href{http://git-scm.com/book}{git-scm.com/book}
\end{frame}

\subsection{Eclipse}
\begin{frame}[<+->]
  \frametitle{Getting Eclipse}
  \begin{block}{GNU/Linux}
	Install eclipse from your package manager. On Debian based systems, that's: \texttt{\# aptitude install eclipse}
  \end{block}
  \begin{block}{Windows}
	\begin{enumerate}
	  \item Download \emph{Eclipse Classic} from \href{http://www.eclipse.org/downloads/}{eclipse.org/downloads}
	  \item Unzip the eclipse folder to \textit{Program Files}
	  \item Create desktop and start menu shortcuts to \texttt{eclipse.exe}
	\end{enumerate}
  \end{block}
\end{frame}

\begin{frame}[<+->]
  \frametitle{Installing the Plug-Ins}
  \begin{itemize}
	\item Eclipse plug-ins can be installed from the marketplace:
	  
	  \texttt{Help > Eclipse Marketplace}
	\item The required plug-ins are:
	  \begin{description}
		\item[EGit] for git support
		\item[Mylyn] for bugtracking and task management
		\item[Connector] GitHub Connector for Mylyn
	  \end{description}
  \end{itemize}
  \begin{block}{Note}
  	After restarting Eclipse you should be able to access the new Git and Task management perspectives:

	\begin{visibleenv}<6>
	  \includegraphics{perspectives}
	\end{visibleenv}
  \end{block}
\end{frame}

\section{Usage}

\subsection{EGit}
\begin{frame}
EGit provides the following context menu:

\fbox{\includegraphics[width=0.7\textwidth]{team-context-menu}}
\end{frame}

\begin{frame}
  The following new perspective is also available:

  \fbox{\includegraphics[width=\textwidth]{git-perspective}}
\end{frame}

\subsection{Mylyn}
\begin{frame}
  To use Mylyn you will first need to add a task repository:

  \fbox{\includegraphics[width=\textwidth]{add-task-repo}}
\end{frame}

\begin{frame}
  You can then add issues which will be synced to GitHub's servers:

  \fbox{\includegraphics[width=\textwidth]{add-issue}}
\end{frame}

\subsection{GitHub}
\begin{frame}
  Now issues added in Eclipse will be added in the GitHub web interface:

  \fbox{\includegraphics[width=0.95\textwidth]{github-issue-overview}}
\end{frame}

\begin{frame}
  We can also comment and manage issues on GitHub the same way as in Eclipse:

  \fbox{\includegraphics[width=0.7\textwidth]{github-issue-details}}
\end{frame}

\begin{frame}
  \begin{center}
	{\bfseries \Huge The End}
  \end{center}
\end{frame}

\end{document}


